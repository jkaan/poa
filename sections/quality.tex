% Begin - Kwaliteitsbewaking hoofdstuk

\section{Kwaliteitsbewaking}
De kwaliteit van mijn stage is te onder verdelen in twee categoriën. Je hebt de categorie met de documenten en dan mijn daadwerkelijke stage bij ConnectSB. De kwaliteit van de documenten wordt grootendeels bepaald door de HZ, zij hebben een aantal regels waar aan gehouden moet worden als er een plan van aanpak en een portfolio opgeleverd moet worden. Deze moeten voldoen aan de opzet die te vinden is in het boek van Succesvol studeren, communiceren en onderzoeken - Alfabetisch naslagwerk voor het hoger onderwijs (2012) van N. van Glabbeek.

Mijn documenten zullen worden beoordeeld aan de hand van naslagwerk van N. van Glabbeek. Deze documenten zullen dus ook gemaakt worden aan de hand van deze naslagwerken om te voldoen aan de eisen van het HZ. In mijn Plan van Aanpak zijn er leerdoelen bepaald en deze moeten S.M.A.R.T. zijn om hiermee zo goed mogelijk mijn stage te kunnen beoordelen.

De kwaliteit van mijn stage bij ConnectSB wordt natuurlijk volledig bepaald door mij. Er zijn een aantal feedback momenten waarmee ik dus kan zien of ik goed bezig ben of niet, maar de kwaliteit wordt bepaald door mijzelf. Door goede leerdoelen op te stellen wordt er een voorbeeld gegeven van de kwaliteit van mijn stage en ik zal mij hier de gehele stage ook aan houden.

ConnectSB heeft zelf ook een aantal kwaliteitseisen die nagelopen moeten worden voordat iets echt af is. Dit zijn de volgende:
\begin{itemize}
\item Unit tests & acceptatie- tests schrijven
\item Twee keer usability tests uitvoeren en hierbij alle feedback verwerken
\end{itemize}

\clearpage

% Eind - Kwaliteitsbewaking hoofdstuk
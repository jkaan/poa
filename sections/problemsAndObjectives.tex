% Begin - Probleemstelling & doelstelling hoofdstuk

\section{Probleemstelling \& Doelstelling}
Dit hoofdstuk zal mijn doelstellingen voor mijn stage bij ConnectSB bevatten. Deze doelstellingen, vanaf nu leerdoelen genoemd, zullen mij helpen om richting te geven aan mijn stage. Deze leerdoelen ga ik ook gebruiken om aan te tonen dat ik daadwerkelijk wat geleerd heb tijdens mijn tijd bij ConnectSB. Van deze vijf leerdoelen was er al één opgesteld door de HZ en de rest heb ik zelf geformuleerd.
Elk leerdoel hoort bij een bepaald onderwerp waarin ik mijzelf wil verdiepen en tevens mijn leerdoel op wil baseren. 
Elk hoofdstuk heeft dus de naam van het onderwerp waar het leerdoel op gebaseerd is.

\subsection{Leren Professioneel Functioneren}
Bij dit leerdoel, die ik zelf niet heb opgesteld zal er geen motivatie aanwezig zijn. Ik zal overigens wel uitleggen welke bewijslast ik hiervoor zal opleveren in mijn portfolio. 
Om aan te tonen dat ik er toe in staat ben om professioneel te functioneren binnen een bedrijf zal ik eerst een definitie geven van wat ik denk dat professioneel functioneren omvat. Als professional in een bedrijf moet je kunnen samenwerken met andere developers, maar ook met designers en business-gerichte personen. Je moet risico's durven nemen. Als een manager je iets opdraagt, maar jij denkt dat het veel mooier zal zijn als je het net iets anders doet moet je dit durven. Je zult als developer vaak alleen aan het programmeren zijn. Het zelfstandig werken en het zelf indelen van je tijd zodat alle taken op tijd af komen hoort ook bij de eigenschappen van een professional. Het werk dat opgeleverd wordt moet accuraat gedaan worden en het moet bijna perfect zijn. 
Als laatste is het natuurlijk belangrijk dat je flexibel bent. Als een manager iets op het laatste moment van je eist moet je dit als werknemer uitvoeren, hier moet je natuurlijk flexibel in zijn dat je dit snel klaarmaakt.
Ik moet tijdens mijn stage aan verschillende opdrachten werken en ik zal op deze eerder genoemde kwaliteiten reflecteren tijdens elke opdracht die ik uitvoer.

\subsection{Facebook API}
Het leerdoel is: Facebook API leren door middel van een applicatie te maken die hiervan gebruik maakt door informatie van een pagina van een klant op te halen.

Dit leerdoel valt onder de competentie Realiseren. Deze gaat in op het vergaren van nieuwe kennis met betrekking tot het realiseren van een applicatie. Deze applicatie zal gebruik maken van de Facebook API, de applicatie zal het mogelijk maken voor een klant om een facebook pagina die beheert wordt door de klant te importeren en hiervan zal de likes, comments, fans etc. opgehaald worden. De bewijslast die hiervoor in mijn portfolio zal staan is een volledig functionele app die in een Facebook pagina draait, maar gehost zal zijn op een server van ConnectSB. Deze applicatie zal ook gebruik maken van de Facebook API, welke aspecten van de API gebruikt zullen worden zal in het portfolio vermeld staan. De verdere technische details met betrekking tot het draaien op Facebook zal ook in het portfolio te vinden zijn. ConnectSB stelt mij in staat om dit leerdoel te halen doordat de developers zich het meeste bezig houden met het maken van applicaties die gebruik maken van de Facebook API.

De reden dat ik gekozen heb voor dit leerdoel is, omdat ik meer wil leren over de Facebook API. Facebook is razend populair en er worden veel applicaties en zelfs games gemaakt die gebruik maken van de Facebook API. Dit loopt uit een van het simpele inloggen met een Facebook account tot het toegang krijgen tot alle pagina\'s die een gebruiker leuk vindt. Er is veel om te ontdekken en dat spreekt me heel erg aan, aan de Facebook API. Ik ben naast het functionele aspect van de API ook benieuwd naar het technische aspect.

\subsection{Meteor, een JavaScript framework}
Meteor is een framework gemaakt in JavaScript en het maakt het mogelijk om gemakkelijk de back- en front-end te maken met JavaScript.

Het leerdoel is: Meteor leren kennen door middel van een applicatie te maken die real-time informatie weergeeft over een klant's social media kanalen aan meerdere eind-gebruikers, dit wordt verwezenlijkt door Meteor's collections en het publish-subscripe pattern te gebruiken.

Dit leerdoel valt onder de competentie Realiseren. Het leerdoel gaat in op het aan leren van een framework voor een bekende taal waarmee het realiseren van een applicatie kan worden volbracht. 

De bewijslast die hiervoor in mijn portfolio terecht komt is een functionele applicatie welke gebruik maakt van Meteor. De bewijslast zal bevatten wat ik heb geleerd met betrekking tot Meteor, mijn ervaringen, welke functionaliteit ik heb ontwikkeld aan de applicatie. ConnectSB is sinds een paar weken bezig met de ontwikkeling van een platform waar alle data uit alle verschillende apps voor de verschillende klanten in opgeslagen zullen zijn. Dit platform wordt ontwikkeld met Meteor en ik zal hier ook aan bij dragen.

Ik heb gekozen voor dit leerdoel, omdat JavaScript tegenwoordig heel erg in is. NodeJS maakt het mogelijk om zelfs de backend volledig in JavaScript te maken en Meteor voegt weer een aantal gave dingen toe. Met Meteor kun je bijvoorbeeld een nieuwe versie uitrollen zonder dat er enige down-time is. De gebruiker zal het niet eens merken als ze op dat moment met de applicatie aan het werken zijn. 

\subsection{Versiebeheer}
Het leerdoel is: Versiebeheer volgens de Gitflow workflow gebruiken zodat er altijd een fout vrije en uitrolbare versie van de applicatie beschikbaar is en als er een fout optreedt deze snel weer ongedaan gemaakt kan worden door naar een vorige versie terug te keren.

Dit leerdoel valt onder de competentie Beheren. Dit leerdoel zal ingaan op het gebruik van versiebeheer en met name op het correct en efficiënt gebruik hiervan volgens een bestaande en bewezen workflow. In mijn portfolio zal ik de volgende bewijslast voor dit leerdoel beschrijven; bij alle gemaakte applicaties zal versiebeheer gebruik worden en zullen ook in productie gezet worden. Bij elke applicatie zullen fouten optreden en door het gebruik van versiebeheer volgens de beschreven workflow zal ik bijna meteen een correcte versie tijdelijk kunnen opleveren terwijl ik tegelijkertijd de fout op los. 

ConnectSB gebruikt Git voor versiebeheer en de Gitflow workflow is een branching model dat zorgt dat er snel terug gekeerd kan worden naar een vorige versie en dat er altijd een versie gereed is die geen bugs bevat.

Versiebeheer wordt door elk goed bedrijf wel gebruikt voor het beheer van code van bepaalde software. Overigens zijn er heel veel dingen die je met versiebeheer kunt vergemakkelijken. Ik wil hier meer kennis van op doen en daarom heb ik ook dit leerdoel opgesteld.

\subsection{Functioneel ontwerp}
Het leerdoel is: Functioneel ontwerp opstellen zodat deze makkelijk leesbaar en begrijpbaar is voor een persoon die niets af weet van de te maken software en dat de opdrachtgever gemakkelijk zijn eisen in het functioneel ontwerp terug kan vinden.

Dit leerdoel valt in de competentie Ontwerpen. Het leerdoel gaat over het opzetten van een functioneel ontwerp en gaat dus heel erg diep in op alle aspecten van een functioneel ontwerp. Bij ConnectSB wordt bij elke applicatie een functioneel- en een technisch ontwerp opgesteld. Ik zal daarom voor de applicaties die ik zal gaan maken functionele ontwerpen die gemaakt zijn opnemen in mijn portfolio en hiermee aantonen dat dit leerdoel behaald is.

Contact met de opdrachtgever en de andere stakeholders is cruciaal in elk software ontwikkel proces. Het is belangrijk voor een opdrachtgever dat een functioneel ontwerp duidelijk is zonder dat een developer uitleg hier over hoeft te geven. Als een functioneel ontwerp goed gemaakt en duidelijk is dan zal het minder lang duren voordat er begonnen kan worden en zo wordt er veel tijd en geld bespaard. Daarom wil ik graag mijn kennis van het correct schrijven van een functioneel ontwerp vergroten.

\clearpage

% Eind - Probleemstelling & doelstelling hoofdstuk
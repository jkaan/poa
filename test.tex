\documentclass{article}
\usepackage[utf8]{inputenc}
\usepackage{fancyhdr}

% Lijst met bronnen, deze zal ik later aan het einde toevoegen van dit document
% http://www.carrieretijger.nl/functioneren/professionele-eigenschappen

% Zorgt voor de mooie header en zorgt ervoor dat page numbering links staat
\pagestyle{fancy}
\fancyhf{}
\lhead{Plan of Action}
\lfoot{\thepage}

% Zorgt dat de inhoudsopgave ook zo heet
\renewcommand{\contentsname}{Inhoudsopgave}

\title{Plan van Aanpak}
\author{Joey Kaan}
\date{September 2014}

\begin{document}

% Begin - Omslag pagina

\begin{center}
{\huge Connect Social Business\par}
\end{center}
\thispagestyle{empty}
\clearpage

% Eind - Omslag pagina

% Begin - Titel pagina

\maketitle
\thispagestyle{empty}
\clearpage

% Eind - Titel pagina

% Begin - Inhoudsopgave

\tableofcontents
\thispagestyle{empty}
\clearpage

% Eind - Inhoudsopgave

% Begin - Achtergronden paragraaf

\section{Achtergronden}
Het bedrijf waar ik bij ga stage lopen/stage loop heet Connect Social Business, meestal ConnectSB genoemd. ConnectSB is een startup die in korte tijd een goede naam heeft opgebouwd door bedrijven te ondersteunen bij alles wat te maken heeft met social media. Een aantal van de klanten waar ConnectSB mee werkt zijn:
\begin{itemize}
\item Jumbo
\item Karwei
\item Fruittella
\end{itemize}
Naast het ondersteunen hebben veel bedrijven behoefte aan het opzetten van hun social media aanwezigheid. ConnectSB biedt hiervoor verschillende apps aan die op Facebook draaien en die met als doel hebben om klanten toe te trekken naar het bedrijf via social media. Ook werken ze in samenwerking met Obi4Wan. Dit is een platform en bedrijf dat zich bezig houdt met webcare en social media monitoring. ConnectSB biedt klanten een dienst aan waarbij ze de webcare van een klant volledig tot zich nemen, hierbij gebruiken ze Obi4Wan.

De developers, een 6-koppig team, zijn verantwoordelijk voor het maken van deze apps. Deze apps komen ook met een administratief beheer. Op deze manier kunnen de community managers hier gemakkelijk mee aan de slag zonder dat een developer hier bij nodig is. 

De community managers zijn weer verantwoordelijk voor zoals eerder gezegd het beheren van de verschillende apps voor de klanten, ook maken ze posts die voor de klant dan op hun facebook pagina gepost worden.

Als laatste komen hier natuurlijk de designers bij kijken, deze ondersteunen bij het maken van apps en het maken van posts om te zorgen dat het er goed uitziet voor de klant.

\clearpage

% Eind - Achtergronden paragraaf

% Begin - Probleemstelling & doelstelling paragraaf

\section{Probleemstelling \& Doelstelling}
Om mijn stage bij ConnectSB zo goed mogelijk af te ronden en aan te tonen dat ik daadwerkelijk wat geleerd heb, worden er leerdoelen opgesteld. Één hiervan is geformuleerd door de HZ zelf en vier zijn opgesteld door mijzelf. Elk leerdoel zal een aparte paragraaf hebben in dit hoofdstuk Probleemstelling & Doelstelling.

\subsection{Leren Professioneel Functioneren}
Bij dit leerdoel, die ik zelf niet heb opgesteld zal er geen motivatie aanwezig zijn. Ik zal overigens wel uitleggen welke bewijslast ik hiervoor zal opleveren in mijn portfolio. 
Om aan te tonen dat ik er toe in staat ben om professioneel te functioneren binnen een bedrijf zal ik eerst een definitie geven van wat ik denk dat professioneel functioneren omvat. Als professional in een bedrijf moet je kunnen samenwerken met andere developers, maar ook met designers en business-gerichte personen. Je moet risico's durven nemen. Als een manager je iets opdraagt, maar jij denkt dat het veel mooier zal zijn als je het net iets anders doet moet je dit durven. Je zult als developer vaak alleen aan het programmeren zijn. Het zelfstandig werken en het zelf indelen van je tijd zodat alle taken op tijd af komen hoort ook bij de eigenschappen van een professional. Het werk dat opgeleverd wordt moet accuraat gedaan worden en het moet bijna perfect zijn. 
Als laatste is het natuurlijk belangrijk dat je flexibel bent. Als een manager iets op het laatste moment van je eist moet je dit als werknemer uitvoeren, hier moet je natuurlijk flexibel in zijn dat je dit snel klaarmaakt.
Ik moet tijdens mijn stage aan verschillende opdrachten werken en ik zal op deze eerder genoemde kwaliteiten reflecteren tijdens elke opdracht die ik uitvoer.

% Eind - Probleemstelling & doelstelling paragraaf

\end{document}
